%%%%%%%%%%%%%%%%%
% This is an sample CV template created using altacv.cls
% (v1.1.3, 30 April 2017) written by LianTze Lim (liantze@gmail.com). Now compiles with pdfLaTeX, XeLaTeX and LuaLaTeX.
% 
%% It may be distributed and/or modified under the
%% conditions of the LaTeX Project Public License, either version 1.3
%% of this license or (at your option) any later version.
%% The latest version of this license is in
%%    http://www.latex-project.org/lppl.txt
%% and version 1.3 or later is part of all distributions of LaTeX
%% version 2003/12/01 or later.
%%%%%%%%%%%%%%%%

%% If you need to pass whatever options to xcolor
\PassOptionsToPackage{dvipsnames}{xcolor}

%% If you are using \orcid or academicons
%% icons, make sure you have the academicons 
%% option here, and compile with XeLaTeX
%% or LuaLaTeX.
% \documentclass[10pt,a4paper,academicons]{altacv}

%% Use the "normalphoto" option if you want a normal photo instead of cropped to a circle
% \documentclass[10pt,a4paper,normalphoto]{altacv}

\documentclass[10pt,a4paper]{altacv}

%% AltaCV uses the fontawesome and academicon fonts
%% and packages. 
%% See texdoc.net/pkg/fontawecome and http://texdoc.net/pkg/academicons for full list of symbols.
%% 
%% Compile with LuaLaTeX for best results. If you
%% want to use XeLaTeX, you may need to install
%% Academicons.ttf in your operating system's font 
%% folder.


% Change the page layout if you need to
\geometry{left=1cm,right=9cm,marginparwidth=6.8cm,marginparsep=1.2cm,top=1.25cm,bottom=1.25cm,footskip=2\baselineskip}

% Change the font if you want to.

% If using pdflatex:
\usepackage[utf8]{inputenc}
\usepackage[T1]{fontenc}
\usepackage[default]{lato}
\usepackage{hyperref}

% If using xelatex or lualatex:
% \setmainfont{Lato}

% Change the colours if you want to
\definecolor{Black}{HTML}{000000}
\definecolor{SlateGrey}{HTML}{2E2E2E}
\definecolor{LightGrey}{HTML}{666666}
\definecolor{greencustom}{HTML}{45540E}
\colorlet{heading}{greencustom}
\colorlet{accent}{Black}
\colorlet{emphasis}{SlateGrey}
\colorlet{body}{LightGrey}

% Change the bullets for itemize and rating marker
% for \cvskill if you want to
\renewcommand{\itemmarker}{{\small\textbullet}}
\renewcommand{\ratingmarker}{\faCircle}

%% sample.bib contains your publications
\addbibresource{sample.bib}

\begin{document}
\name{Brendon A. Kay}
\tagline{Python Enthusiast - Software Development}
%% \photo{2.8cm}{Globe_High}
\personalinfo{%
  % Not all of these are required!
  % You can add your own with \printinfo{symbol}{detail}
  \email{bak04280 at gmail dot com}
  \location{Burlington, VT}
  \homepage{\href{https://brendonakay.github.io}{brendonakay.github.io}}
  \linkedin{\href{https://www.linkedin.com/in/brendon-kay-39298366}{brendon-kay-39298366}}
  \github{\href{https://github.com/brendonakay/}{brendonakay}}
  %% You MUST add the academicons option to \documentclass, then compile with LuaLaTeX or XeLaTeX, if you want to use \orcid or other academicons commands.
%   \orcid{orcid.org/0000-0000-0000-0000}
}

%% Make the header extend all the way to the right, if you want. 
\begin{fullwidth}
\makecvheader
\end{fullwidth}

%% Provide the file name containing the sidebar contents as an optional parameter to \cvsection.
%% You can always just use \marginpar{...} if you do
%% not need to align the top of the contents to any
%% \cvsection title in the "main" bar.
\cvsection[page1sidebar]{Experience}

\cvevent{Application Support Engineer}{Logic Supply, Inc.}{November 2017 -- Present}{South Burlington, VT}
\begin{itemize}
\item Maintained production systems of Logic Supply's ERP, website, and internal tooling
services. Extended and implemented new modules in the ERP and e-commerce website to meet custom business requirements. Implemented in-house auditing tools for analytics and fault prevention by way of popular Python packages; such as Pandas, Requests, BeutifulSoup, and SQLAlchemy.
Maintained Google business tools, such as Google Scripts, for business intelligence reporting.
\end{itemize}

\divider

\cvevent{QA Engineer}{Logic Supply, Inc.}{April 2017 -- November 2017}{South Burlington, VT}
\begin{itemize}
\item Ensureed code quality for the business’s e-commerce website and ERP system through
manual testing methods, user acceptance testing, and by reviewing and
updating unit tests written by myself and the development team. Maintained the development GitLab CI for automated and scheduled test runs and code quality checks. Wrote automated functional tests in Selenium for automated website testing.
\end{itemize}

\divider

\cvevent{QA Engineer}{Allscripts Healtchare Solutions, Inc.}{September 2016 -- March 2017}{South Burlington, VT}
\begin{itemize}
\item Maintained quality of the TouchWorks EHR product by means of automated testing.
Daily activities included writing Selenium scripts, as well as coordinating user
acceptance testing. Research requirements for products and solutions. Regular
Scrum meetings in a fully Agile work environment.
\end{itemize}

\divider

\cvevent{Support Engineer}{Union Street Media}{February 2015 -- March 2016}{Burlington, VT}
\begin{itemize}
\item End to end website, domain, and email support for USM's many clients. Maintained and
updated their proprietary CMS platform, built on a LAMP stack. Daily activities included
updating PHP code in the development Git repository to apply a fix for a bug, configuring DNS for a
client, as well as maintaining the MySQL databases.
\end{itemize}

\cvsection{Projects}

\cvevent{}{\href{https://github.com/brendonakay/computer-manufacturing-network}{Computer Manufacturing DLT Network}}{February 2018 -- Ongoing}{}
A Hyperledger Composer business network that functions as a transparent and immutable Supply Chain
ecosystem for manufacturing computers using open source blockchain technology:
\smallskip
\begin{itemize}
\item Hyperledger Composer
\item Node.js
\item Git
\end{itemize}

\medskip

% \cvsection{A Day of My Life}
% 
% Adapted from @Jake's answer from http://tex.stackexchange.com/a/82729/226
% \wheelchart{outer radius}{inner radius}{
% comma-separated list of value/text width/color/detail}
% \wheelchart{1.5cm}{0.5cm}{%
%   3/6em/accent!30/Sports \& Hobbies,
%   8/8em/accent!80/Work,
%   1/8em/accent!10/Transport,
%   2/10em/accent!20/Personal IT projects,
%   8/8em/accent!80/Sleep,
%   2/6em/accent!20/Cooking \& Eating
% }

%% If the NEXT page doesn't start with a \cvsection but you'd
%% still like to add a sidebar, then use this command on THIS
%% page to add it. The optional argument lets you pull up the 
%% sidebar a bit so that it looks aligned with the top of the
%% main column.
% \addnextpagesidebar[-1ex]{page3sidebar}

\end{document}
